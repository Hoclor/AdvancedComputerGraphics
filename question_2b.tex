%!TEX TS-program = xelatex
%!TEX encoding = UTF-8 Unicode

\documentclass[11pt]{article}
\usepackage[UKenglish]{babel}
\usepackage[UKenglish]{isodate} 
\usepackage[margin=2cm, a4paper]{geometry}
\usepackage{graphicx, fancyhdr}
\usepackage{fontspec, xltxtra, xunicode}
\usepackage{enumitem, tabu}

\setlist{noitemsep}

\usepackage{array, xcolor}
\definecolor{lightgray}{gray}{0.8}
\newcolumntype{L}{>{\raggedleft}p{0.16\textwidth}}
\newcolumntype{R}{p{0.8\textwidth}}
\newcommand\VRule{\color{lightgray}\vrule width 0.5pt}

\pagestyle{fancy}
\lhead{} % nothing
\chead{pbqk24}
\rhead{} % nothing
\lfoot{} % nothing
\cfoot{\thepage}
\rfoot{} % nothing

\title{\vspace{-6.0ex}Advanced Computer Graphics Assignment}
\author{pbqk24}
\date{10 December 2018}

\begin{document}
\maketitle

\section*{Execution Instructions}
To execute the set of Python scripts, make sure they are all placed in the same directory (as Questions 2 and 4 re-use the code from Questions 1 and 3, respectively). The scripts were written and tested with Python 3.7.1, but any version should work. The scripts also require that the geomdl, numpy, and matplotlib libraries be available.\\

The scripts for Questions 1 and 4 display the requested output, then launch into an interactive script that allows you to customize the displayed output. Questions 2 and 3 simply display the output and then exit. Most scripts also output some text regarding the output and/or how to interact with the script.

\section*{Question 2b}

In order to support proper shape editing of the heart-shaped object through the decomposed Bezier curves, as performed for Question 2a, the continuity between the Bezier curve segments must be maintained. This ensures that the produced curve/object visually looks good, and that there are no cracks, sudden extreme angles, or gaps (that are not intended). The property is applied by maintaining $C^n$ continuity, for some $n$. This means that the direction and magnitude of the derivative of the curve equations at the point where they join are equal through the $n\textsuperscript{th}$ derivative.\\

As $C^n$ continuity is computationally expensive to maintain, instead $G^1$ continuity is maintained in practice. This is applied by ensuring that the directions of the two segments' tangent vectors are equal at the point that they join up (differing from $C^1$ continuity by not placing a requirement on the magnitude of the tangent vectors). This is done as it achieves acceptable continuity, while being much less expensive to compute and maintain.


\end{document}